\let\negmedspace\undefined
\let\negthickspace\undefined
%\RequirePackage{amsmath}
\documentclass[journal,12pt,twocolumn]{IEEEtran}
 \usepackage[utf8]{inputenc}
 \usepackage{graphicx}
 \usepackage{amsmath}
 \usepackage{mathrsfs}
\usepackage{txfonts}
\usepackage{stfloats}
\usepackage{bm}
\usepackage{cite}
\usepackage{cases}
\usepackage{subfig}
 \usepackage{amsfonts}
 \usepackage{amssymb}
 \usepackage{enumitem}
\usepackage{mathtools}
\usepackage{tikz}
\usepackage{circuitikz}
\usepackage{verbatim}
\usepackage[breaklinks=false,hidelinks]{hyperref}
\usepackage{listings}
\usepackage{calc}
\usepackage{float}
\usepackage{longtable}
\usepackage{multirow}
\usepackage{multicol}
\usepackage{color}
\usepackage{array}
\usepackage{hhline}
\usepackage{ifthen}
\usepackage{chngcntr}

\newcommand{\BEQA}{\begin{eqnarray}}
\newcommand{\EEQA}{\end{eqnarray}}
\newcommand{\define}{\stackrel{\triangle}{=}}
\bibliographystyle{IEEEtran}
%\bibliographystyle{ieeetr}
\def\inputGnumericTable{}
\let\vec\mathbf
\providecommand{\pr}[1]{\ensuremath{\Pr\left(#1\right)}}
\providecommand{\sbrak}[1]{\ensuremath{{}\left[#1\right]}}
\providecommand{\lsbrak}[1]{\ensuremath{{}\left[#1\right.}}
\providecommand{\rsbrak}[1]{\ensuremath{{}\left.#1\right]}}
\providecommand{\brak}[1]{\ensuremath{\left(#1\right)}}
\providecommand{\lbrak}[1]{\ensuremath{\left(#1\right.}}
\providecommand{\rbrak}[1]{\ensuremath{\left.#1\right)}}
\providecommand{\cbrak}[1]{\ensuremath{\left\{#1\right\}}}
\providecommand{\lcbrak}[1]{\ensuremath{\left\{#1\right.}}
\providecommand{\rcbrak}[1]{\ensuremath{\left.#1\right\}}}
%\providecommand{\abs}[1]{\left\vert#1\right\vert}
\providecommand{\res}[1]{\Res\displaylimits_{#1}}
\newcommand{\myvec}[1]{\ensuremath{\begin{pmatrix}#1\end{pmatrix}}}
\newcommand{\mydet}[1]{\ensuremath{\begin{vmatrix}#1\end{vmatrix}}}
\newcommand{\question}{\noindent \textbf{Question: }}
\newcommand{\solution}{\noindent \textbf{Solution: }}
\newcommand{\note}{\noindent \textbf{Note: }}
\newcommand{\tofind}{\noindent \textbf{To find: }}
\newcommand{\given}{\noindent \textbf{Given: }}
\newcommand{\final}{\noindent \textbf{Final Answer: }}
\title{Assignment 4}
\author{Mayuri Chourasia\\BT21BTECH11001}
\date{}
\begin{document}
% make the title area
\maketitle
\question Two coins (a one rupee coin and a two rupee coin) are tossed at once. Find the sample space.\\

\solution \\

\given A pair of two distinguishable coins.\\

\tofind The sample space of the when the two coins are tossed simultaneously,

Let us denote the events of tossing one coin and two coins  simultaneously by random variables $X$ and $Y$ respectively.\\
\\
Let Sample space of $X$ be $S_x$,where:
\begin{align}
    &S_x=\cbrak{H,T}
\end{align}
Where, H denotes Head and T denotes Tail.\\
\\
Let Sample space of $Y$ be $S_y$ and we know that $S_y$ will be the Cartesian product of $S_x$ with itself:\\
Such that:
\begin{align}
    &S_y = S_x \times S_x
\end{align}
So,the sample for tossing two coins at a time can be given as:\\ 
$S_y$ = {(x,y) : $x$ is the outcome of the first coin and $y$ is the outcome of the second coin}\\
And thus,
\begin{align}
    S_y = \cbrak{(H,H),(H,T),(T,H),(T,T)}
\end{align}
\end{document}
