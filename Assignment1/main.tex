\documentclass[ twocoloumns, 20pts]{IEEEtran}
\begin{document}
\title{\huge{Probability and Random Variables}\\{Assignment 1}}
\author{{Mayuri Rajesh Chourasia}\\
{BT21BTECH11001}\\
\vspace{5mm}
Question 8(B)\\
ICSE 2019 QUESTION PAPER\\
\fontsize{11}{1}\usefont{T1}{cmr}{m}{n}The vertices of a triangle ABC are A(3,8), B(-1,2) and C (6,-6). Find:\\
\put(100,10){(i) Slope of BC\\}
 (ii) Equation of a line perpendicular to BC and passing through A.\\
\vspace{5mm}
\put{ANSWER:}}
\maketitle
\section*{\large{PART(1)}}
\large{Let $(x_2,y_2)$be the co-ordinates of point B(-1,2),\\
So,\\
   $x_2=-1$\\
   $y_2= 2$\\
Let $(x_3,y_3)$ be the co-ordinates of point C(6,-6),\\
So,\\
   $x_3= 6$\\
   $y_3= -6$\\}
   
To find the slope between two points, we use the
slope point formula. For two points, $(x_2,y_2)$ and
$(x_3,y_3)$, the point-slope formula is given by:\\

Slope = ${\frac{(y_3-y_2)}{(x_3 -x_2)}}$\\

Therefore, the slope of line BC will be given as:\\

Slope = $\frac{(-6)-2)}{(6-(-1))}$\\

\hspace{1.1cm}=$\frac{-8}{7}$\\

\framebox{    The slope of line BC is $\frac{-8}{7}$.   }\\
\vspace{40mm}

\section*{\large{PART(2)}}\\
{
Let slope of line perpendicular to BC be Slope(2), and
let slope of line BC be Slope(1)\\
we know that,\\
Slope(2) = - $\frac{1}{Slope(1}$\\
}

\hspace{1.3cm}=$\frac{1}{\frac{-8}{7}}$\\

\hspace{1.3cm}= 7/8\\

In general case, for a line passing through a point
$(x_1,y_1)$ and having a slope m can be\\ 
given by the equation :   $( y - y_1) = m ( x - x_1)$\\

Therefore, the equation of a line passing through\\
The point A(3,8) and having a slope of  7/8 will be\\
given as :  ( y – 8 ) = 7/8( x – 3 )\\

\hspace{1.5cm}8( y - 8 )= 7(x – 3 )\\

\hspace{1.5cm}8y – 64 = 7x - 21\\

\hspace{1.5cm}8y – 7x = 43\\

\framebox{The equation of line perpendicular to BC and}

\framebox{\hspace{4.7mm}passing through A(3,8) is  8y – 7x = 43 \hspace{4.7mm}}

\end{document}
