\let\negmedspace\undefined
\let\negthickspace\undefined
%\RequirePackage{amsmath}
\documentclass[journal,12pt,twocolumn]{IEEEtran}
 \usepackage[utf8]{inputenc}
 \usepackage{graphicx}
 \usepackage{amsmath}
 \usepackage{amsfonts}
 \usepackage{amssymb}
 \usepackage{enumitem}
\usepackage{mathtools}
\usepackage[breaklinks=false]{hyperref}
\usepackage{listings}
\usepackage{calc}

\newcommand{\BEQA}{\begin{eqnarray}}
\newcommand{\EEQA}{\end{eqnarray}}
\newcommand{\define}{\stackrel{\triangle}{=}}
\bibliographystyle{IEEEtran}
%\bibliographystyle{ieeetr}
\let\vec\mathbf
\newcommand{\myvec}[1]{\ensuremath{\begin{pmatrix}#1\end{pmatrix}}}
\newcommand{\mydet}[1]{\ensuremath{\begin{vmatrix}#1\end{vmatrix}}}
\newcommand{\question}{\noindent \textbf{Question: }}
\newcommand{\solution}{\noindent \textbf{Solution: }}
\title{Probability and Random Variables\\Assignment 1}
\author{Mayuri Chourasia\\BT21BTECH11001}
\date{}
\begin{document}
% make the title area
\maketitle
\question
\begin{enumerate}[label=]
\item The vertices of a triangle ABC are A(3,8), B(-1,2) and C (6,-6). Find:
\begin{enumerate}
    \item Slope of BC
    \item Equation of a line perpendicular to BC and passing through A
\end{enumerate}
\end{enumerate}
\solution
\begin{enumerate}
\item Let $\vec{A}, \vec{B}, \vec{C}$ be the points vectors.
	\begin{align}
		\vec{A} = \myvec{3 \\ 8} ,
		\vec{B} = \myvec{-1 \\ 2}  ,
		\vec{C} = \myvec{6 \\ -6}
	\end{align}
	$\therefore$ The direction vector of $BC$ is,
    \begin{align}
    &\vec{d} = \vec{C} - \vec{B}
    \\
	&\vec{d} = \myvec{6 \\ -6} - \myvec{-1 \\ 2}
	\\
	&\vec{d} = \myvec{7 \\ -8}
    \end{align}
    We know that, if the direction vector of a line is represented by a matrix \vec{d}=\myvec{d_1\\d_2} then the slope for the same can be represented by $(\frac{d_2}{d_1})$.\\
    \medskip\\
    Therefore in this case the slope can be given as:
    \begin{align}
        m=\frac{-8}{7}
    \end{align}
    \textbf{The slope of BC is $\frac{-8}{7}$}
    \medskip\\
\item Now, let \textbf{z} be the normal vector of the line BC, hence we know that;
    \begin{align}
	\vec{d}^{\top}\vec{z} &= 0
	\\
	\myvec{7 & -8}\vec{z} &= 0
	\\
	\vec{z} &= \myvec{8 \\ 7}
	\\
	\vec{z}^{\top} &= \myvec{8 & 7}
    \end{align}
     Let L be the line that passes through A and is perpendicular to BC, and let \textbf{n} be the normal vector of the line L, in such a case we can say that;
     \begin{align}
	\vec{z}^{\top}\vec{n} &= 0
	\\
	\myvec{8 & 7}\vec{n} &= 0
	\\
	\vec{n} &= \myvec{7 \\ -8}
	\\
	\vec{n}^{\top} &= \myvec{7 & -8}
    \end{align}
    The normal equation of the line L is given by, 
    \begin{align}
	\vec{n}^{\top}(\vec{X} - \vec{A}) &= 0
	\\
	\myvec{7 & -8}(\myvec{x \\ y} - \myvec{3 \\ 8})&= 0
	\\
	\myvec{7 & -8}\myvec{x - 3 \\ y - 8} &= 0
    \end{align}
    \textbf{Thus, line L $\equiv \myvec{7 & -8}\myvec{x - 3 \\ y - 8} = 0$}
\end{enumerate}
\end{document} 
