\documentclass[journal,12pt,twocolumn]{IEEEtran}
\usepackage{amsmath}
\usepackage{tcolorbox}
\usepackage{amssymb}
\usepackage{amsthm}

\begin{document}
     \title{Probability and Random Variables\\Assignment 1}
     \author{Mayuri Chourasia$^{*}$\\BT21BTECH11001}
     \maketitle
     \section*{\textbf{Question 8 (B)\\ICSE 2019 PAPER}}
     The vertices of a triangle ABC are A(3,8), B(-1,2) and C (6,-6). Find:\\
     (i) Slope of BC\\
     (ii) Equation of a line perpendicular to BC and passing through A.
     \section*{\textbf{ANSWER}}
     \subsection*{PART 1:}
     
     Let $(x_2,y_2)$ be the co-ordinates of point B(-1,2),
     So,\\
     $x_2=-1$\\
     $y_2= 2$\\
     
     Let $(x_3,y_3)$ be the co-ordinates of point C(6,-6),\\
     So,\\
     $x_3= 6$\\
     $y_3= -6$\\
    
     To find the slope between two points, we use the
     slope point formula. For two points, $(x_2,y_2)$ and
     $(x_3,y_3)$, the point-slope formula is given by:\\
     \[\text{Slope} = \frac{y_2-y_1}{x_2-x_1}\ \]
     Therefore, the slope of line BC will be given as:\\
     \begin{align*}
         \text{Slope} &= \frac{(-6)-2)}{(6-(-1))}\\
         &=\frac{-8}{7}
     \end{align*}
     \medskip
     \begin{tcolorbox}
     \begin{center}
         {\textbf{The slope of line BC is -8/7.}}
     \end{center}
     \end{tcolorbox}
     \subsection*{PART 2:}
     Let slope of line perpendicular to BC be Slope(2), and let slope of line BC be Slope(1)\\
     we know that,\\
     \begin{align*}
         \text{Slope(2)} &= -\frac{1}{\text{{Slope(1)}}}\\
         &=-\frac{\text{1}}{(-8/7)}\\
         &=\frac{7}{8}\\
     \end{align*}
     In general case, for a line passing through a point $(x_1,y_1)$ and having a slope m can be given by the equation :\\
     \begin{equation*}
         (y - y_1) = m ( x - x_1)
     \end{equation*}
     \medskip\\
     Therefore, the equation of a line passing through the point A(3,8) and having a slope of  7/8 will be given as :\\
     \begin{align*}
         ( y – 8 ) &= 7/8( x – 3 )\\
         8( y - 8 )&= 7(x – 3 )\\
         8y – 64 &= 7x - 21\\
         8y – 7x &= 43\\
     \end{align*}
     \begin{tcolorbox}
     \begin{center}
          \textbf{{The equation of line perpendicular to B and passing through A(3,8) is\\  8y – 7x = 43}}
     \end{center}
     \end{tcolorbox}
\end{document}
