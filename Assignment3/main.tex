\let\negmedspace\undefined
\let\negthickspace\undefined
%\RequirePackage{amsmath}
\documentclass[journal,12pt,twocolumn]{IEEEtran}
 \usepackage[utf8]{inputenc}
 \usepackage{graphicx}
 \usepackage{amsmath}
 \usepackage{mathrsfs}
\usepackage{txfonts}
\usepackage{stfloats}
\usepackage{bm}
\usepackage{cite}
\usepackage{cases}
\usepackage{subfig}
 \usepackage{amsfonts}
 \usepackage{amssymb}
 \usepackage{enumitem}
\usepackage{mathtools}
\usepackage{tikz}
\usepackage{circuitikz}
\usepackage{verbatim}
\usepackage[breaklinks=false,hidelinks]{hyperref}
\usepackage{listings}
\usepackage{calc}
\usepackage{float}
\usepackage{longtable}
\usepackage{multirow}
\usepackage{multicol}
\usepackage{color}
\usepackage{array}
\usepackage{hhline}
\usepackage{ifthen}
\usepackage{chngcntr}

\newcommand{\BEQA}{\begin{eqnarray}}
\newcommand{\EEQA}{\end{eqnarray}}
\newcommand{\define}{\stackrel{\triangle}{=}}
\bibliographystyle{IEEEtran}
%\bibliographystyle{ieeetr}
\def\inputGnumericTable{}

\let\vec\mathbf

\providecommand{\pr}[1]{\ensuremath{\Pr\left(#1\right)}}
\providecommand{\sbrak}[1]{\ensuremath{{}\left[#1\right]}}
\providecommand{\lsbrak}[1]{\ensuremath{{}\left[#1\right.}}
\providecommand{\rsbrak}[1]{\ensuremath{{}\left.#1\right]}}
\providecommand{\brak}[1]{\ensuremath{\left(#1\right)}}
\providecommand{\lbrak}[1]{\ensuremath{\left(#1\right.}}
\providecommand{\rbrak}[1]{\ensuremath{\left.#1\right)}}
\providecommand{\cbrak}[1]{\ensuremath{\left\{#1\right\}}}
\providecommand{\lcbrak}[1]{\ensuremath{\left\{#1\right.}}
\providecommand{\rcbrak}[1]{\ensuremath{\left.#1\right\}}}
\providecommand{\abs}[1]{\left\vert#1\right\vert}
\providecommand{\res}[1]{\Res\displaylimits_{#1}}
\newcommand{\myvec}[1]{\ensuremath{\begin{pmatrix}#1\end{pmatrix}}}
\newcommand{\mydet}[1]{\ensuremath{\begin{vmatrix}#1\end{vmatrix}}}

\newcommand{\question}{\noindent \textbf{Question: }}
\newcommand{\solution}{\noindent \textbf{Solution: }}
\newcommand{\note}{\noindent \textbf{Note: }}
\newcommand{\final}{\noindent \textbf{Final Answer: }}


\title{Assignment 3}
\author{Mayuri Chourasia\\BT21BTECH11001}
\date{}
\begin{document}
% make the title area
\maketitle
\question There are 40 students in Class X of a school of whom 25 are girls and 15
are boys. The class teacher has to select one student as a class representative. She
writes the name of each student on a separate card, the cards being identical. Then
she puts cards in a bag and stirs them thoroughly. She then draws one card from the
bag.\\
Find the probability such that:
\begin{enumerate}[label=(\roman{enumi})]
	\item the name written on the card is the name of a girl
	\item the name written on the card is the name of a boy
\end{enumerate}
\solution We obtain the following distribution of students according to the question:
\begin{table}[H]
\input{studentstats.tex}
	\caption{Distribution of Students}
	\label{tab1}
\end{table}
Let's denote the outcome of the experiment by a random variable $X$ such that $X\in \cbrak{0 , 1}$. \\
where,
\begin{table}[H]
\input{table2.tex}
	\caption{Randomn Variable and Event Distribution}
	\label{tab2}
\end{table}
\begin{enumerate}[label=(\roman{enumi})]
\item The probability that the name written on the card is the name of a girl can be given as :
\begin{align}
	 &\pr{X=0}\\
	 &=\frac{\text{Number of girl students in the class}}{\text{Total number of students in the class}}\\
	&=\frac{25}{40}\\
	&=\fbox{0.625}
\end{align}
\final The probability that the name written on the card is the name of a girl is 0.625.
\item The probability that the name written on the card is the name of a boy can be given as :
\begin{align}
	 &\pr{X=1}\\
	 &=\frac{\text{Number of boy students in the class}}{\text{Total number of students in the class}}\\
	&=\frac{15}{40}\\
	&=\fbox{0.375}
\end{align}
\note Since we know that the event mentioned are mutually exclusive and exhaustive in nature, the probability that the name written on the card is the name of a boy can also be given as:
\begin{align}
	\pr{X=1}&=1-\pr{X=0}\\
	&=1-0.625\\
	&=\fbox{0.375}
\end{align}
\final The probability that the name written on the card is the name of a boy is 0.375.
\end{enumerate}
\end{document}
