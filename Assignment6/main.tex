\let\negmedspace\undefined
\let\negthickspace\undefined
%\RequirePackage{amsmath}
\documentclass[journal,12pt,twocolumn]{IEEEtran}
 \usepackage[utf8]{inputenc}
 \usepackage{graphicx}
 \usepackage{amsmath}
 \usepackage{mathrsfs}
\usepackage{txfonts}
\usepackage{stfloats}
\usepackage{bm}
\usepackage{cite}
\usepackage{tikz}
\usetikzlibrary{trees,arrows}
\usepackage{cases}
\usepackage{subfig}
 \usepackage{amsfonts}
 \usepackage{amssymb}
 \usepackage{enumitem}
\usepackage{mathtools}
\usepackage{circuitikz}
\usepackage{verbatim}
\usepackage[breaklinks=false,hidelinks]{hyperref}
\usepackage{listings}
\usepackage{calc}
\usepackage{float}
\usepackage{longtable}
\usepackage{multirow}
\usepackage{multicol}
\usepackage{color}
\usepackage{array}
\usepackage{hhline}
\usepackage{ifthen}
\usepackage{chngcntr}
\usepackage{times}
\usepackage{ulem}
\usepackage[nocenter]{qtree}
\usepackage{tree-dvips}
\usepackage{gb4e}
\newcommand{\BEQA}{\begin{eqnarray}}
\newcommand{\EEQA}{\end{eqnarray}}
\newcommand{\define}{\stackrel{\triangle}{=}}
\bibliographystyle{IEEEtran}
%\bibliographystyle{ieeetr}
\def\inputGnumericTable{}
\let\vec\mathbf
\providecommand{\pr}[1]{\ensuremath{\Pr\left(#1\right)}}
\providecommand{\sbrak}[1]{\ensuremath{{}\left[#1\right]}}
\providecommand{\lsbrak}[1]{\ensuremath{{}\left[#1\right.}}
\providecommand{\rsbrak}[1]{\ensuremath{{}\left.#1\right]}}
\providecommand{\brak}[1]{\ensuremath{\left(#1\right)}}
\providecommand{\lbrak}[1]{\ensuremath{\left(#1\right.}}
\providecommand{\rbrak}[1]{\ensuremath{\left.#1\right)}}
\providecommand{\cbrak}[1]{\ensuremath{\left\{#1\right\}}}
\providecommand{\lcbrak}[1]{\ensuremath{\left\{#1\right.}}
\providecommand{\rcbrak}[1]{\ensuremath{\left.#1\right\}}}
%\providecommand{\abs}[1]{\left\vert#1\right\vert}
\providecommand{\res}[1]{\Res\displaylimits_{#1}}
\newcommand{\myvec}[1]{\ensuremath{\begin{pmatrix}#1\end{pmatrix}}}
\newcommand{\mydet}[1]{\ensuremath{\begin{vmatrix}#1\end{vmatrix}}}
\newcommand{\question}{\noindent \textbf{Question: }}
\newcommand{\solution}{\noindent \textbf{Solution: }}
\newcommand{\note}{\noindent \textbf{Note: }}
\newcommand{\tofind}{\noindent \textbf{To find: }}
\newcommand{\given}{\noindent \textbf{Given: }}
\newcommand{\final}{\noindent \textbf{Final Answer: }}
\title{Assignment 6}
\author{Mayuri Chourasia\\BT21BTECH11001}
\date{}
\begin{document}
% make the title area
\maketitle
\question Consider the experiment of tossing a coin. If the coin shows head, toss it
again but if it shows tail, then throw a die. Find the
conditional probability of the event that ‘the die shows
a number greater than 4’ given that ‘there is at least
one tail’.\\
\solution \\
The outcomes of the experiment can be
represented in following diagrammatic manner called
the ‘tree diagram’.\\
% Set the overall layout of the tree
\tikzstyle{level 1}=[level distance=2cm, sibling distance=4cm]
\tikzstyle{level 2}=[level distance=2cm, sibling distance=1cm]

% Define styles for bags and leafs
\tikzstyle{bag} = [text width=4em, text centered]
\tikzstyle{end} = [circle, minimum width=3pt,fill, inner sep=0pt] 
\begin{tikzpicture}[grow=right, sloped]
\node[bag] {}
    child {
        node[bag] {Tail(T)}        
            child {
                node[end, label=right:
                    {(T,1)}] {}
                edge from parent
            }
            child {
                node[end, label=right:
                    {(T,2)}] {}
                edge from parent
            }
             child {
                node[end, label=right:
                    {(T,3)}] {}
                edge from parent
            }
             child {
                node[end, label=right:
                    {(T,4)}] {}
                edge from parent
            }
             child {
                node[end, label=right:
                    {(T,5)}] {}
                edge from parent
            }
             child {
                node[end, label=right:
                    {(T,6)}] {}
                edge from parent
            }
            edge from parent 
    }
    child {
        node[bag] {Head(H)}        
        child {
                node[end, label=right:
                    {(H,H)}] {}
                edge from parent
            }
            child {
                node[end, label=right:
                    {(H,T)}] {}
                edge from parent
            }
        edge from parent         
    };
\end{tikzpicture}\\

The sample space of the experiment may be
described as\\
    S = \{(H,H), (H,T), (T,1), (T,2), (T,3), (T,4), (T,5), (T,6)\}\\
Thus, the probabilities assigned to the 8 elementary
events (H, H), (H, T), (T, 1), (T, 2), (T, 3) (T, 4), (T, 5), (T, 6) are $\frac{1}{4}$,$\frac{1}{4}$,$\frac{1}{12}$,$\frac{1}{12}$,$\frac{1}{12}$,$\frac{1}{12}$,$\frac{1}{12}$,$\frac{1}{12}$ respectively which is clear from the below given tree diagram.
\tikzstyle{level 1}=[level distance=2cm, sibling distance=4cm]
\tikzstyle{level 2}=[level distance=2cm, sibling distance=1cm]

% Define styles for bags and leafs
\tikzstyle{bag} = [text width=4em, text centered]
\tikzstyle{end} = [circle, minimum width=3pt,fill, inner sep=0pt] 
\begin{tikzpicture}[grow=right, sloped]
\node[bag] {}
    child {
        node[bag] {Tail(T)}        
            child {
                node[end, label=right:
                    {(T,1)}] {}
                edge from parent
                node[above] {$\frac{1}{12}$}
            }
            child {
                node[end, label=right:
                    {(T,2)}] {}
                edge from parent
                node[above] {$\frac{1}{12}$}
            }
             child {
                node[end, label=right:
                    {(T,3)}] {}
                edge from parent
                node[above] {$\frac{1}{12}$}
            }
             child {
                node[end, label=right:
                    {(T,4)}] {}
                edge from parent
                node[above] {$\frac{1}{12}$}
            }
             child {
                node[end, label=right:
                    {(T,5)}] {}
                edge from parent
                node[above] {$\frac{1}{12}$}
            }
             child {
                node[end, label=right:
                    {(T,6)}] {}
                edge from parent
                node[above] {$\frac{1}{12}$}
            }
            edge from parent
                node[above] {$\frac{1}{2}$} 
    }
    child {
        node[bag] {Head(H)}        
        child {
                node[end, label=right:
                    {(H,H)}] {}
                edge from parent
                node[above] {$\frac{1}{4}$}
            }
            child {
                node[end, label=right:
                    {(H,T)}] {}
                edge from parent
                node[above] {$\frac{1}{4}$}
            }
        edge from parent
        node[above] {$\frac{1}{2}$}
    };
\end{tikzpicture}\\
Let F be the event that ‘there is at least one tail’ and E be the event ‘the die shows a number greater than 4’. \\
Then\\
\begin{align}
    F &= {(H,T), (T,1), (T,2), (T,3), (T,4), (T,5), (T,6)}\\
E &= {(T,5), (T,6)} \text{ and } E\cap F = {(T,5), (T,6)}
\end{align}
Now\\
\begin{align}
    P(F) &=\sum_{i=1}^{6}{P({(T,i)})}\\
    &=\frac{1}{4}+\frac{1}{4}+\frac{1}{12}+\frac{1}{12}+\frac{1}{12}+\frac{1}{12}+\frac{1}{12}+\frac{1}{12}=\frac{3}{4}
\end{align}
And
\begin{align}
    P(E\cap F)a &= P ({(T,5)}) + P ({(T,6)})\\ &=\frac{1}{12}+\frac{1}{12}=\frac{1}{6}
\end{align} 
hence
\begin{align}
    P(E|F)=\frac{P(E\cap F)}{P(F)}=\frac{\frac{1}{6}}{\frac{3}{4}}=\frac{2}{9}
\end{align}
\end{document}
