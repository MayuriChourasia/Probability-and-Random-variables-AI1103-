\let\negmedspace\undefined
\let\negthickspace\undefined
%\RequirePackage{amsmath}
\documentclass[journal,12pt,twocolumn]{IEEEtran}
 \usepackage[utf8]{inputenc}
 \usepackage{graphicx}
 \usepackage{amsmath}
 \usepackage{mathrsfs}
\usepackage{txfonts}
\usepackage{stfloats}
\usepackage{bm}
\usepackage{cite}
\usepackage{cases}
\usepackage{subfig}
 \usepackage{amsfonts}
 \usepackage{amssymb}
 \usepackage{enumitem}
\usepackage{mathtools}
\usepackage{tikz}
\usepackage{circuitikz}
\usepackage{verbatim}
\usepackage[breaklinks=false,hidelinks]{hyperref}
\usepackage{listings}
\usepackage{calc}
\usepackage{float}
\usepackage{longtable}
\usepackage{multirow}
\usepackage{multicol}
\usepackage{color}
\usepackage{array}
\usepackage{hhline}
\usepackage{ifthen}
\usepackage{chngcntr}

\newcommand{\BEQA}{\begin{eqnarray}}
\newcommand{\EEQA}{\end{eqnarray}}
\newcommand{\define}{\stackrel{\triangle}{=}}
\bibliographystyle{IEEEtran}
%\bibliographystyle{ieeetr}
\def\inputGnumericTable{}
\let\vec\mathbf
\providecommand{\pr}[1]{\ensuremath{\Pr\left(#1\right)}}
\providecommand{\sbrak}[1]{\ensuremath{{}\left[#1\right]}}
\providecommand{\lsbrak}[1]{\ensuremath{{}\left[#1\right.}}
\providecommand{\rsbrak}[1]{\ensuremath{{}\left.#1\right]}}
\providecommand{\brak}[1]{\ensuremath{\left(#1\right)}}
\providecommand{\lbrak}[1]{\ensuremath{\left(#1\right.}}
\providecommand{\rbrak}[1]{\ensuremath{\left.#1\right)}}
\providecommand{\cbrak}[1]{\ensuremath{\left\{#1\right\}}}
\providecommand{\lcbrak}[1]{\ensuremath{\left\{#1\right.}}
\providecommand{\rcbrak}[1]{\ensuremath{\left.#1\right\}}}
%\providecommand{\abs}[1]{\left\vert#1\right\vert}
\providecommand{\res}[1]{\Res\displaylimits_{#1}}
\newcommand{\myvec}[1]{\ensuremath{\begin{pmatrix}#1\end{pmatrix}}}
\newcommand{\mydet}[1]{\ensuremath{\begin{vmatrix}#1\end{vmatrix}}}
\newcommand{\PROBLEM}{\noindent \textbf{Problem: }}
\newcommand{\solution}{\noindent \textbf{Solution: }}
\newcommand{\note}{\noindent \textbf{Note: }}
\newcommand{\tofind}{\noindent \textbf{To find: }}
\newcommand{\given}{\noindent \textbf{Given: }}
\title{Assignment 7}
\author{MAYURI CHOURASIA\\BT21BTECH11001}
\date{}
\begin{document}
% make the title area
\maketitle
\PROBLEM Let A and B be independent events with P(A) = 0.3 and P(B) = 0.4. Find:\\
\begin{enumerate}
    \item $P(A \cap B)$
    \item $P(A \cup B)$
    \item $P(A|B)$
    \item $P(B|A)$
\end{enumerate}
\solution \\
\given \begin{enumerate} 
    \item P(A)=0.3
    \item P(B)=0.4
\end{enumerate}
Let us denote event A by X=0 and event B by X=1, where X is a Random Variable.\\
so we have,\\
\begin{align}
    &P(X=0)=0.3\\
    &P(X=1)=0.4
\end{align}
\begin{enumerate}
\item \begin{align}
    P(X=0 \cap X=1) &= P(X=0)\times P(X=1)\\
    &= 0.3 \times 0.4\\
    &= 0.12
\end{align}

\item \begin{align}
P(X=0 \cup X=1) &= P(X=0) + P(X=1) - P(X=0 \cup X=1)\\
&= 0.3 + 0.4 - 0.12\\
&=0.58
\end{align}
\item \begin{align}
    P(X=0 | X=1) &= \frac{P(X=0 \cap X=1)}{P(X=1)}\\
&= \frac{0.12}{0.4}\\
&=0.3
\end{align}
\item \begin{align}
    P(X=1 | X=0) &= \frac{P(X=0 \cap X=1)}{P(X=0)}\\
&= \frac{0.12}{0.3}\\
&=0.4
\end{align}
\end{enumerate}
\end{document}
